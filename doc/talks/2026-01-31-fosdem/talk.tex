%\nonstopmode
\documentclass[aspectratio=169]{beamer}
\usepackage[utf8]{inputenc}
% \usepackage[frenchb]{babel}
\usepackage{amsmath}
\usepackage{mathtools}
\usepackage{breqn}
\usepackage{multirow}
\usetheme{boxes}
\usepackage{graphicx}
%\useoutertheme[footline=authortitle,subsection=false]{miniframes}

\beamertemplatenavigationsymbolsempty

\definecolor{TitleOrange}{RGB}{255,137,0}
\setbeamercolor{title}{fg=TitleOrange}
\setbeamercolor{frametitle}{fg=TitleOrange}

\definecolor{ListOrange}{RGB}{255,145,5}
\setbeamertemplate{itemize item}{\color{ListOrange}$\blacktriangleright$}

\definecolor{verygrey}{RGB}{70,70,70}
\setbeamercolor{normal text}{fg=verygrey}


\usepackage{tabu}
\usepackage{multicol}
\usepackage{vwcol}
\usepackage{stmaryrd}
\usepackage{graphicx}

\usepackage[normalem]{ulem}

\title{Garage Object Storage: 2.0 update and best practices}
\subtitle{a new storage platform for self-hosted geo-distributed clusters}
\author{Maximilien Richer, Deuxfleurs}
\date{FOSDEM '26}

\begin{document}

\begin{frame}
	\centering
	\includegraphics[width=.3\linewidth]{../../sticker/Garage.pdf}
	\vspace{1em}

	{\large\bf Maximilien Richer, Deuxfleurs}
	\vspace{1em}

	\url{https://garagehq.deuxfleurs.fr/}
	Matrix channel: \texttt{\#garage:deuxfleurs.fr}
\end{frame}

\begin{frame}
	\frametitle{Our objective at Deuxfleurs}

	\begin{center}
	    French association promoting digital sovereignty and privacy\\
		through self-hosting hosting \textbf{as an alternative to large cloud providers}
	\end{center}
	\vspace{2em}

	\vspace{2em}
	\begin{center}
		\textbf{This requires \underline{resilience}}\\
		{\footnotesize (we want good uptime/availability with low supervision)}
	\end{center}
\end{frame}

\begin{frame}
	\frametitle{But what is Garage, exactly?}
	\textbf{Garage is a self-hosted drop-in replacement for the Amazon S3 object store}\\
	\vspace{.5em}
	that implements resilience through geographical redundancy on commodity hardware
		\begin{center}
			\includegraphics[width=.8\linewidth]{assets/garageuses.png}
		\end{center}
\end{frame}

\begin{frame}
	\frametitle{What makes Garage different?}
	\textbf{Coordination-free:}
			\vspace{2em}
	\begin{itemize}
		\item No Raft or Paxos
			\vspace{1em}
		\item Internal data types are CRDTs
			\vspace{1em}
		\item All nodes are equivalent (no master/leader/index node)
	\end{itemize}
			\vspace{2em}
	$\to$ less sensitive to higher latencies between nodes
\end{frame}

\begin{frame}
	\frametitle{What makes Garage different?}
		\begin{center}
		    TODO update with latest garage and minio versions
			\includegraphics[width=.9\linewidth]{assets/endpoint-latency-dc.png}
		\end{center}
\end{frame}

\begin{frame}
	\frametitle{What makes Garage different?}
	\textbf{Consistency model:}
			\vspace{2em}
	\begin{itemize}
		\item Not ACID (not required by S3 spec) / not linearizable
			\vspace{1em}
		\item \textbf{Read-after-write consistency}\\
			{\footnotesize (stronger than eventual consistency)}
	\end{itemize}
\end{frame}

\begin{frame}
	\frametitle{What makes Garage different?}
	\textbf{Location-aware:}
	\vspace{2em}
		\begin{center}
			\includegraphics[width=\linewidth]{assets/location-aware.png}
		\end{center}
	\vspace{2em}
	Garage replicates data on different zones when possible
\end{frame}

\begin{frame}
	\frametitle{What makes Garage different?}
		\begin{center}
			\includegraphics[width=.8\linewidth]{assets/map.png}
		\end{center}
\end{frame}

\begin{frame}
	\frametitle{An ever-increasing compatibility list}
		\begin{center}
			\includegraphics[width=.7\linewidth]{assets/compatibility.png}
		\end{center}
\end{frame}

\begin{frame}
	\frametitle{Version history and roadmap}
	\begin{itemize}
		\item v0.3: initial beta release (2021)
		\item v0.7: first released version (2022)
		\item v1.0: stable release (2024), will be deprecated in summer 2026 1y after v2.0 was released
        \item v2.0: stable release (2025)
        \begin{itemize}
        \item new HTTP admin API
        \item reworded replication configuration: \texttt{replication\_mode} changed to \texttt{replication\_factor} \& \texttt{consistency\_policy}
        \end{itemize}
        \item
	\end{itemize}
	\begin{center}
	    v3.0: TBA may include versionning support, tag on buckets and objets, retention policies...
	\end{center}
\end{frame}

\begin{frame}
	\centering
	{\large\bf Best practices for Garage deployments}
\end{frame}

\begin{frame}
	\frametitle{Things you should know}
	\begin{itemize}
		\item no TLS support, use your own proxy
		\item no anonymous access (use website endpoint)
		\item you need to assign roles to nodes manually
        \item the replication factor cannot be changed easily
        \item the default region is \texttt{garage} and not \texttt{us-east-1}
        \item only use the \texttt{degraded} consistency policy for data recovery!
	\end{itemize}
\end{frame}

\begin{frame}
	\frametitle{What hardware should I use?}
	\begin{itemize}
	    \item do NOT use network file storage (NFS, SMB, etc.) for \texttt{\/metadata}
		\item get a \textbf{write-intensive flash disk} for the \texttt{\/metadata} folder
		\item set \texttt{metadata} on a RAID1 if possible, with a COW filesystem (e.g. Btrfs or ZFS)
		\item get large HDDs for the \texttt{\/data} folder
		\item use XFS and garage multi-hdd mode for best performance
		\item you can use a RAID for data but you'll leave a lot of performance on the table
	\end{itemize}
	\center\textit{Garage doesn't require a powerful CPUs nor much RAM, but your performance will depend on your disks!}
\end{frame}

\begin{frame}
	\frametitle{Picking a metadata engine}
	All files-to-block mappings are stored in the metadata engine, including bucket and object metadata. Files below 3KB are stored directly in the metadata engine.
    \vspace{1em}
	\begin{itemize}
        \item Sled: removed in 1.x, move to SQLite or LMDB
	    \item \textbf{SQLite}: safer, \textbf{recommended for small clusters and single-node}
        \item LMDB: faster, recommended for large clusters with metadata redundancy
        \begin{itemize}
            \item Warning: limited to 480 bytes per key with LMDB (not an issue in practice)
        \end{itemize}
        \item Fjall: experimental but promising rust-native engine, test it and let us know!
	\end{itemize}
	\center{Metadata engine can be set node per node, and changed later with a migration tool}
\end{frame}

\begin{frame}
	\frametitle{Single-node deployment}
	\begin{itemize}
		\item garage was initially designed for multi-node deployments
		\item single-node deployments are possible, but you will lose resilience
		\item \textbf{If you do please ensure you have backups} (especially for metadata)
		\begin{itemize}
            \item set up \texttt{metadata\_auto\_snapshot\_interval}
        \end{itemize}
		\item use sqlite to minimize data loss risks on powercuts
		\item or use a UPS!
	\end{itemize}
	\vspace{1em}
	Use \texttt{github.com/bikeshedder/garage-single-node} for an easy single-node setup!
\end{frame}

\begin{frame}
	\frametitle{Multi-node deployment}
	\begin{itemize}
		\item try to have geo-distributed zones
		\item multiple nodes per zone to add more capacity
		\item at least 3 zones for best resilience
		\item keep in mind your available network and IO bandwidth
		\item \textbf{Rebalancing a cluster can take multiple weeks with large HDDs and slow network links}
		\item monitor your nodes with Prometheus + Grafana
	\end{itemize}
	\center{Deuxfleurs has been running a 9TB (3TB usable) 8-nodes cluster (3+3+2) over retail fiber (10ms site-to-site latency) for close to 5 years now. We heard there are petabyte clusters out there!}
\end{frame}

\begin{frame}
	\frametitle{Deploying and administering garage at scale}
	\begin{itemize}
		\item deploy with your favorite tool (eg. Ansible) and system manager (eg. systemd)
		\item or use Docker, docker-compose, Kubernetes or Nomad
		\item Kubernetes and Consul are supported for node-to-node discovery
		\begin{itemize}
		\item you'll still have to manage the layout manually!
        \end{itemize}
		\item use gateway nodes to optimize network usage
		\item ajust \texttt{resync-tranquility} and \texttt{scrub-tranquility} to your ressources
	\end{itemize}
	\center{Kubernetes storage controller: \texttt{github.com/bmarinov/garage-storage-controller}}
\end{frame}

\begin{frame}
	\frametitle{Community UI available!}
	\begin{center}
			\includegraphics[width=0.9\linewidth]{assets/community-ui.png}\\
			\vspace{-1em}
		\url{https://github.com/khairul169/garage-webui}
	\end{center}
\end{frame}

\begin{frame}
	\frametitle{Official Embedded UI comming later this year!}
	\begin{center}
	    \includegraphics[width=0.9\linewidth]{assets/Garage Web Admin - Dashboard@2x.png}\\
			\vspace{-1em}
	\end{center}
\end{frame}

\begin{frame}
	\frametitle{Official Embedded UI comming this year!}
	\begin{center}
	    \includegraphics[width=0.9\linewidth]{assets/Garage Web Admin - Bucket details page@2x.png}\\
			\vspace{-1em}
	\end{center}
\end{frame}

\begin{frame}
	\frametitle{How to make sense of garage metrics?}
	\begin{center}
	    \includegraphics[width=0.7\linewidth]{assets/garage-stats.png}\\
			\vspace{-1em}
	\end{center}
\end{frame}

\begin{frame}
	\frametitle{What if things go wrong?}
	\begin{itemize}
	    \item set logs to debug with \texttt{RUST_LOG=garage_api_common=debug,garage_api_s3=debug,garage=debug}
		\item auth issues: check your reverse proxy configuration
		\item slow resync: check your network and disk IO usage, and \texttt{resync-tranquility} worker configuration
		\item big LMDB database: stop garage and compact with \texttt{mdb\_copy -c}
		\item ask us on matrix \texttt{\#garage:deuxfleurs.fr} or open an issue on git.deuxfleurs.fr!
		\begin{itemize}
		\item provide the output of \texttt{garage status}, \texttt{garage stats} and relevant metrics and logs
        \end{itemize}
	\end{itemize}
\end{frame}

\begin{frame}
	\frametitle{Moving from Minio}
	\begin{itemize}
		\item list your buckets and your keys
		\item create buckets and keys on the garage cluster
		\begin{itemize}
		\item you cannot import non-garage keys yet, patch to come soon!
		\end{itemize}
		\item loop over buckets, copy with rclone
		\begin{itemize}
		\item see doc \url{https://garagehq.deuxfleurs.fr/documentation/connect/cli/}
		\end{itemize}
		\item blog post coming soon!
	\end{itemize}
\end{frame}

\begin{frame}
	\frametitle{Demo time!}
\end{frame}

\begin{frame}
	\frametitle{Get Garage now!}
	\begin{center}
			\includegraphics[width=.3\linewidth]{../../logo/garage_hires.png}\\
			\vspace{-1em}
		\url{https://garagehq.deuxfleurs.fr/}\\
		Matrix channel: \texttt{\#garage:deuxfleurs.fr}

		\vspace{2em}
			\includegraphics[width=.09\linewidth]{assets/rust_logo.png}
			\includegraphics[width=.2\linewidth]{assets/AGPLv3_Logo.png}
	\end{center}
\end{frame}

\end{document}

%% vim: set ts=4 sw=4 tw=0 noet spelllang=fr :
